\documentclass{article}
\usepackage[utf8]{inputenc}

\title{Elaboration}
\author{jack.mathieson }
\date{August 2019}

\begin{document}

\maketitle

\tableofcontents
\section{Introduction}
    This Elaboration is meant to outline the goals, requirements and possible technical processes of my future MRes Thesis. Pains, gains and possible solutions have been posited in two earlier scoping papers. 

\section{Thesis Vision}

    The thesis I plan to write will be on narrative construction by Iranian Refugees living in Sydney. It will examine the ways narratives are expressed and built by these people and will take an emic perspective, attempting to introduce the topic from their own view of it. The primary method used in Anthropology to capture such a perspective is Ethnography, which utilizes extensive participant-observation and relational, qualitative research rather than quantitative analysis. Words and their significance, context and it's situation, agency and it's ownership: it is this kind of data and meta-data that will be most useful, so a system of analysis that can cope with data of this nature is necessary.
    
\section{ Scoping Results}

The results of the past scoping papers identified an area where significant time-constraints are placed on the ethnographic process: transcribing interviews. As interviews are (one of) the best methods of capturing complex, qualitative data they will be used frequently throughout the research process. The disadvantage of this approach is that for every hour spent interviewing a participant there are at \textit{least} three hours of transcribing, often more. This takes up valuable time that might otherwise be spent in participant-observation, which is possibly the primary way
Ethnography is conducted.

As such, scoping revealed that the most useful tool for ethnographic research that involves multiple interviews would be an audio recording tool that can perform the following:
\begin{enumerate}
\item Record and segment a continuous stream of sound.
\item Provide a means of storing meta-data for each recorded stream.
\item Shift and re-organize selected segments of sound while retaining a copy of the original stream.
\item Be able to identify sounds that correspond to words and transcribe these.
\item The recorder must be able to record a specific, non-audible signal that can be delivered on command and recorded alongside the stream of sound. The connection may be via blue-tooth or any form of wireless connection provided the recorder can record it alongside the audible sound-stream*.
\end{enumerate}

\textbf{*Note on Scoping point 5}: this requirement is for the purpose of identifying a key moment in an interview. The ideal tool would be a small "button" that the interviewer can soundlessly click at any point in the interview when something particularly meaningful had been said. If this soundless signal is recorded alongside the audio-stream then the interviewer need only listen to the segment of recording containing that signal rather than scroll through the whole recording for "that moment" they remember hearing.

In addition to this, the tool would need to have a simple way of transferring the recorded file from the audio recorder to a computer. Alternatively if the physical audio recorder has built-in software or downloadable software that can be installed on a computer this would also be sufficient.

\section{Incompatible Tools}
A simple recording of a stream of sound is insufficient, so most standard Windows Media programs would not help. While such programs can accurately capture a single stream of sound and even cut and shuffle segments of the recording, they cannot meet the requirements listed in points 2, 4 and 5.

Additionally, a tool that can record and transcribe is still not enough. Even if the level of accuracy is insufficient, as is usually the case for systems like the YouTube transcription tool, one sitting of listening to the recording will pick up on such mistakes. The key here is that even with a pin-point accurate transcription tool, like Transcribe, such a system is still only designed to pick up on \textit{audible} data, which does not meet point 5.

Furthermore, a tool that can record audible  \textit{and} an inaudible, electronic signals rarely has transcribing capabilities as well. A software used in music recording and creation might have the capacity for picking up on an inaudible Bluetooth signal, and would be perfect as a visual tool that can reorganize and copy different streams, however it still would not solve the issue of discerning words from sound and writing them down.

And finally there is the issue of a convenient, simple file-transfer process from recorder to computer. A useful feature of the Roland R-26 Digital Field Audio Recorder is that it comes equipped with a USB port, making file-transfer simple. Unfortunately this does not solve the issue of point 5, as the recorder, while being able to record multiple \textit{audible} streams, cannot pick up on non-audible signals.

\section{Possible Tool Application}
As outlined before, the ideal tool would be a system that can record both streams of sound and inaudible signals, reorganize and transcribe those streams. One potential way of bringing all of these facets together would be to have two different tools working together: a recorder and a transcriber software. The recorder must be able to pick up on audible sounds and inaudible signals, and whatever format it then presents them in on a computer must include both the audible and inaudible recordings. However the Transcribing tool need only transcribe the audible sounds, not the inaudible signals. Thus, the only requirement of the Transcribing tool is that it be a software compatible with whatever format the recording tool delivers the audio file in to the computer.

\section{Concluded Plan}
In line with the above requirements and possible applications,  this elaboration desires the following tools: 
\begin{enumerate}
\item An audio recorder of the same quality used in music creation which can record inaudible, wireless signals
\item A transcriber software that is compatible with the audio-file format produced by the audio recorder.
\end{enumerate}




\end{document}
